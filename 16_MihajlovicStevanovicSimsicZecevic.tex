% !TEX encoding = UTF-8 Unicode

\documentclass[a4paper]{article}

\usepackage{color}
\usepackage{url}
\usepackage[T2A]{fontenc} % enable Cyrillic fonts
\usepackage[utf8]{inputenc} % make weird characters work
\usepackage{graphicx}

\usepackage[english,serbian]{babel}
%\usepackage[english,serbianc]{babel} %ukljuciti babel sa ovim opcijama, umesto gornjim, ukoliko se koristi cirilica

\usepackage[unicode]{hyperref}
\hypersetup{colorlinks,citecolor=green,filecolor=green,linkcolor=blue,urlcolor=blue}

%\newtheorem{primer}{Пример}[section] %ćirilični primer
\newtheorem{primer}{Primer}[section]

\begin{document}

\title{Medijska pismenost\\ \small{Seminarski rad u okviru kursa\\Tehničko i naučno pisanje\\ Matematički fakultet}}

\author{Ljiljana Mihajlović, Danica Šimšić,\\ Ana Stevanović, Tijana Zečević\\ kontakt email adresa autora}
\date{11.~novembar 2022.}
\maketitle

\abstract{
Gotovo neprimetno, mediji su se iz sredstva informisanja transformisali u sredstvo informisanja \emph{i} manipulacije. Javnost je ;esto nedovoljno informisana. Povremena predavanja stručnih lica u osnovnim školama na temu zaštite podataka na interentu i retke edukacije starijih generacija o ispravnom tumačenju medijskog sadržaja su površna rešenja. Iščitavanjem naučnih radova i sajtova za mlade došlo se do zaključka da je jedino rešenje ovog problema aktivno razvijanje medijske pismenosti, upoznavanjem sa načinom na koji mediji funkcionišu, raznim vidovima manipulacije i razvijanjem kritičkog mišljenja. 


Ključne reči : savremeni mediji, društveni mediji, medijska manipulacija, spinovanje, dezinformacije, kritičko mišljenje, medijska pismenost
}

\tableofcontents

\newpage

\section{Uvod}
\label{sec:uvod}

Razvoj tehnologije doveo je do intenzivnog prisustva medija u životu svakog pojednca i jačanja medijskog uticaja na formiranje javnog mnjenja o društvenoj i političkoj stvarnosti. Gotovo neprimetno, mediji su iz sredstva informisanja prerasli u sredstvo informisanja i manipulacije. Mnoštvo informacija sa kojima smo svakodnevno u kontaktu, dobijeno iz različitih izvora i plasirano na različite načine, oblikuje javno mnjenje o društvenoj i političkoj stvarnosti.


Sedamdesetih godina prošlog veka UNESKO prepoznaje potrebu da se u obrazovne sisteme širom sveta uvede upoznavanje učenika sa osnovnim elementima i karakteristikama medija koji ih okružuju. 


Pojam medijske pismenosti uveden je 1992. godine na konferenciji o istoj (\emph{National Leadereship Conference on Media Literacy, 1992}), i definisan kao „sposobnost pristupa, analize, vrednovanja i slanja poruka posredstvom medija“ . Jednostavno rečeno, medijska pismenost je sposobnost da se objektivno, racionalno i kritičkim razmišljanjem prepozna \textbf{istinita}, \textbf{potpuna} i \textbf{nepristrasna informacija}.  
Stoga je najpre važno objasniti način funkcionisanja medija i značaj njihovog uticaja u savremenom društvu, da bi se razumela suština medijske pismenosti.

\newpage

\section{Tradicionalni i društveni mediji}

\section{Spinovanje i manipulacija}
\label{sec:naslovN}


Za razvoj medijske pismenosti najpre se treba upitati ko proizvodi medijske poruke (država, pojedinac, izvesna medijska kuća…) i zašto ih proizvodi. Koji je cilj te medijske poruke? Kakva reakcija se očekuje i da li se određena informacija plasira u datom trenutku upravo zbog očekivane reakcije? 

 	U ovom poglavlju upoznajemo se sa terminom spinovanja.\textbf{ Spinovanje} je savremeni termin za vid medijske manipulacije koji obuhvata skup propagandnih trikova. Karakteristične metode spina podrazumevaju preuveličavanje izvesnih događaja i selektivno i tempirano iznošenje činjenica u javnost, najčešće u cilju preusmeravanja pažnje sa loših vesti i važnih problema na nevažne.

Činjenice iz nezvaničnih izvora se pojavljuju u medijima sa ciljem da izazovu burnu reakciju, zatim se poriču, pa opet tvrde, čime se manipuliše i novinarima. Sa sličnom namerom stvaraju se nepostojeći, često vrlo uznemirujući problemi kako bi se pored njih „provukli“ stvarni problemi, odnosno, kako bi ih javnost lakše prihvatila. 

Sposobnost razlikovanja činjenica od nečijeg mišljenja pomaže u razvijanju kritičkih i analitičkih sposobnosti, i svesti da nije sve uvek onako kao što izgleda. Dok se činjenicom smatra nešto što se može dokazati kao istinito, mišljenje se odnosi na lično uverenje i kako se neko oseća u vezi sa nečim. 
Većina metoda spina zasniva se na manipulaciji našim emocijama. Medijske poruke koje dolaze do nas imaju moć da nas navedu da osećamo tugu, uznemirenost, iznenađenje, ljutnju, radost, (nacionalni) ponos. Dr Den Sigel, naučnik u oblasti neurobiologije, razvio je princip „Prepoznaj i obuzdaj“. Da bi se izbegla nagla reakcija i mogućnost da poverujemo u lažnu informaciju, važno je da zastanemo i pokušamo da definišemo emociju koju je medijska poruka izazvala. Gledano sa aspekta svrhe medijskog sadržaja, jedino je izveštavanje vid informisanja. Svi ostali sadržaji su vid ubeđivanja (tabela \ref{tab:tabela1}). 

Pošto je danas netačna informacija lako proverljiva, osmišljena je nova, mnogo podmuklija metoda manipulacije. Često se istovremeno plasiraju proverene informacije, poluistine i  dezinformacije, i na svima se insistira istim intenzitetom, na takav način da običan čovek počinje da veruje da su stvari ili mnogo bolje, ili mnogo gore nego što jesu, ali više nije siguran ni ko ga je u to ubedio, ni zašto u to veruje.

\begin{table}[h!]
\begin{center}
\caption{ Svrha medijske poruke u odnosu na sadržaj.}
\begin{tabular}{|c|c|p{3cm} |} \hline
Vrsta sadržaja:& Vid:& Svrha:\\ \hline
Izveštavanje&Informisanja&Da definiše\\ \hline
Mišljenje &Ubeđivanja&Da utiče (na ono što verujete)\\ \hline
Oglašavanje &Ubeđivanja&Da utiče (na ono što kupujete)\\ \hline
Društveno oglašavanje &Ubeđivanja&Da utiče na to kako ćete da se ponašate\\ \hline
Pr &Ubeđivanja&Da utiče na to kako ćete da razmišljate o neko kompaniji\\ \hline
Propaganda &Ubeđivanja&Da utiče ili nametne neki politički sta, izbor itd.\\ \hline
\end{tabular}
\label{tab:tabela1}
\end{center}
\end{table}


\section{Dezinformacije}

\section{Medijska Pismenost}





\section{Zaključak}
\label{sec:zakljucak}

Medijska pismenost je važna zato što nas uči da tumačimo različite vrste medija, interpretiramo i razumemo informacije i njihov izvor postavljajući prava pitanja, razlikujemo činjenice od mišljenja, promišljamo o medijskim porukama sa različitih tačaka gledišta i donosimo ispravne odluke. Pomaže nam da razlikujemo stvarnost od sveta koji su kreirali mediji. 


Važno je pomenuti da mediji bitno utiču na uspostavljanje društvenog sistema vrednosti. Medijska pismenost nas uči da razlučimo kojim medijima treba verovati, pa i tada uvek treba proveriti određenu informaciju, pre nego što je objavimo ili podelimo. Cilj je pronaći istinitu, nepristrasnu i potpunu informaciju u moru medijskog sadržaja.



\addcontentsline{toc}{section}{Literatura}
\appendix

\iffalse
\bibliography{seminarski} 
\bibliographystyle{plain}
\fi

\begin{thebibliography}{9}


\bibitem{prirucnik1} Grupa autora (po azbučnom redu): Vlajković Bojić Violeta, Miladinović Nenad, Milijić Subić Dejana, Milošević Ivana. Priručnik za nastavnike \emph{ Naši učenici u svetu kritičkog mišljenja i medijske pismenosti}. Dostupno na adresi: https://zuov.gov.rs/prirucnik-za-nastavnike-nasi-ucenici-u-svetu-kritickog-misljenja-i-medijske-pismenosti/.

\bibitem{prirucnik2} Nedim Sejdinović i Tatjana Ljubić, \emph{Osnove medijske pismenosti – priručnik za učenike}, 2014. Dostupno na adresi:
http://www.medijskapismenost.net/dokument/Osnove-medijske-pismenosti---prirucnik-za-ucenike/.



\end{thebibliography}


\appendix
\section{Dodatak}
Ovde pišem dodatne stvari, ukoliko za time ima potrebe.
Ovde pišem dodatne stvari, ukoliko za time ima potrebe.
Ovde pišem dodatne stvari, ukoliko za time ima potrebe.
Ovde pišem dodatne stvari, ukoliko za time ima potrebe.
Ovde pišem dodatne stvari, ukoliko za time ima potrebe.


\end{document}

